\begin{center}
  \large\textbf{ABSTRAK}
\end{center}

\addcontentsline{toc}{chapter}{ABSTRAK}

\vspace{2ex}

\begingroup
% Menghilangkan padding
\setlength{\tabcolsep}{0pt}

\noindent
\begin{tabularx}{\textwidth}{l >{\centering}m{2em} X}
  Nama Mahasiswa    & : & \name{}         \\

  Judul Tugas Akhir & : & \tatitle{}      \\

  Pembimbing        & : & 1. \advisor{}   \\
                    &   & 2. \coadvisor{} \\
\end{tabularx}
\endgroup

% Ubah paragraf berikut dengan abstrak dari tugas akhir
Dalam Kompetisi Robot sepak bola beroda, tim IRIS mendapatkan 
prestasi terbaiknya yaitu menjuarai RoboCup peringkat 3. 
Dalam permainan, Robot IRIS menggunakan kamera omnivision untuk 
mendeteksi hal-hal pada lingkungan sekitar. Selama ini, kalibrasi 
pada kamera omnivision menggunakan regresi polinominal satu arah sehingga 
hasilnya kurang baik pada arah yang lainnya. Tugas Akhir ini mengusulkan 
untuk menggunakan metode yang lerbih kompleks yaitu menggunakan 
pendekatan Machine Learning. 

% Ubah kata-kata berikut dengan kata kunci dari tugas akhir
Kata Kunci: Omnivision, Kalibrasi, IRIS
