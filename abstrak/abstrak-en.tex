\begin{center}
  \large\textbf{ABSTRACT}
\end{center}

\addcontentsline{toc}{chapter}{ABSTRACT}

\vspace{2ex}

\begingroup
% Menghilangkan padding
\setlength{\tabcolsep}{0pt}

\noindent
\begin{tabularx}{\textwidth}{l >{\centering}m{3em} X}
  \emph{Name}     & : & \name{}         \\

  \emph{Title}    & : & \engtatitle{}   \\

  \emph{Advisors} & : & 1. \advisor{}   \\
                  &   & 2. \coadvisor{} \\
\end{tabularx}
\endgroup

% Ubah paragraf berikut dengan abstrak dari tugas akhir dalam Bahasa Inggris
\emph{In Soccer Robotics Competition, IRIS team archieved 3rd  
Position in RoboCup. In the game, IRIS Robots used Omnivision to 
sensing their environtment. The current Calibration method is using 
polynomial regression for one direction, so that the other direction 
is not calibrated and give incorrect data. This Final Project propose 
new method that use Machine Learning.}

% Ubah kata-kata berikut dengan kata kunci dari tugas akhir dalam Bahasa Inggris
\emph{Keywords}: \emph{Omnivision}, \emph{Calibration}, \emph{IRIS}
