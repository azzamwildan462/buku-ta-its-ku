\begin{center}
  \Large
  \textbf{KATA PENGANTAR}
\end{center}

\addcontentsline{toc}{chapter}{KATA PENGANTAR}

\vspace{2ex}

% Ubah paragraf-paragraf berikut dengan isi dari kata pengantar

Puji dan syukur kehadirat Allah SWT atas segala rahmat, karunia, dan hidayah-Nya sehingga penulis dapat menyelesaikan penulisan tugas akhir ini.

Penelitian ini disusun dalam rangka memenuhi salah satu syarat untuk menyelesaikan pendidikan di Program Studi \studyprogram{} Fakultas Teknologi Elektro dan Informatika Cerdas, Institut Teknologi Sepuluh Nopember.
Oleh karena itu, penulis mengucapkan terima kasih kepada:

\begin{enumerate}[nolistsep]

  \item Keluarga, Ibu, Bapak dan Saudara tercinta yang telah menyemangati dan memberikan dukungan moral kepada penulis selama menempuh pendidikan di Institut Teknologi Sepuluh Nopember.

  \item Mas Pandu Surya Tantra S.T, M.T, selaku kakak tingkat yang selalu menemani dan memberikan arahan kepada penulis selama mengerjakan tugas akhir ini.

  \item Seseorang wanita yang bisa memotivasi penulis untuk menyelesaikan tugas akhir ini.

\end{enumerate}

Akhir kata, semoga tugas akhir ini dapat memberikan manfaat bagi pembaca dan penulis sendiri. Penulis menyadari bahwa masih banyak kekurangan dalam penulisan tugas akhir ini. Oleh karena itu, penulis mengharapkan kritik dan saran yang membangun dari pembaca demi kesempurnaan tugas akhir ini.

\begin{flushright}
  \begin{tabular}[b]{c}
    \place{}, \MONTH{} \the\year{} \\
    \\
    \\
    \\
    \\
    \name{}
  \end{tabular}
\end{flushright}
