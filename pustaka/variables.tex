% Atur variabel berikut sesuai namanya

% nama
\newcommand{\name}{Azzam Wildan Maulana}
\newcommand{\authorname}{Azzam Wildan Maulana}
\newcommand{\nickname}{Azzam}
\newcommand{\advisor}{Muhtadin, S.T., M.T.}
\newcommand{\coadvisor}{Ahmad Zaini, S.T., M.T.}
\newcommand{\examinerone}{Prof. Dr. Ir. Mauridhi Hery Purnomo, M.Eng.}
\newcommand{\examinertwo}{Eko Pramunanto, S.T., M.T.}
\newcommand{\examinerthree}{Ir. Hany Boedinugroho, M.T.}
\newcommand{\headofdepartment}{Dr. Supeno Susiki Nugroho. S.T., M.T}

% identitas
\newcommand{\nrp}{5024201010}
\newcommand{\advisornip}{19800603 200604 1 003}
\newcommand{\coadvisornip}{19750419 200212 1 003}
\newcommand{\examineronenip}{19580916 198601 1 001}
\newcommand{\examinertwonip}{19661203 199412 1 001}
\newcommand{\examinerthreenip}{19610706 198701 1 001}
\newcommand{\headofdepartmentnip}{19700313 199512 1 001}

% judul
\newcommand{\tatitle}{KALIBRASI KAMERA \emph{OMNIVISION} PADA \emph{MOBILE ROBOT} MENGGUNAKAN \emph{MACHINE LEARNING}}
\newcommand{\engtatitle}{\emph{OMNIVISION CALIBRATION ON MOBILE ROBOT USING MACHINE LEARNING}}

% tempat
\newcommand{\place}{Surabaya}

% jurusan
\newcommand{\studyprogram}{Teknik Komputer}
\newcommand{\engstudyprogram}{Computer Engineering}

% fakultas
\newcommand{\faculty}{Fakultas Teknologi Elektro dan Informatika Cerdas}
\newcommand{\engfaculty}{Fakultas Teknologi Elektro dan Informatika Cerdas}

% singkatan fakultas
\newcommand{\facultyshort}{FTEIC}
\newcommand{\engfacultyshort}{ELECTICS}

% departemen
\newcommand{\department}{Teknik Komputer}
\newcommand{\engdepartment}{Computer Engineering}

% kode mata kuliah
\newcommand{\coursecode}{EC234701}
